\section{Opgave 2}

1. Hvilke af følgende udsagn er sande?

\begin{center}
    1. $\forall x \in \mathbb{N}: \exists y \in \mathbb{N}: x<y$
\\
    2. $\forall x \in \mathbb{N}: \exists ! y \in \mathbb{N}: x<y$
\\
    3. $\exists y \in \mathbb{N}: \forall x \in \mathbb{N}: x<y$
    
\end{center}

1. $\forall x \in \mathbb{N}: \exists y \in \mathbb{N}: x<y$

Dette udsagn er \textbf{sandt}. Det siger at for alle naturlige tal $x$, eksistere der naturlige tal $y$ som er større end $x$. Hvilket er sandt da vi altid kan finde et naturligt tal $y$ som kan være større end $x$.

2. $\forall x \in \mathbb{N}: \exists ! y \in \mathbb{N}: x<y$

Dette udsagn er \textbf{falsk}. Det siger at for alle naturlige tal $x$, eksisterer der lige præcis kun et $y$, sådan at y er større end $x$. Hvilket ikke passer, fordi der findes y $\mathbb{N}$ som kan være større end x. 

3. $\exists y \in \mathbb{N}: \forall x \in \mathbb{N}: x<y$

Dette udsagn er \textbf{falsk}. Dette udsagn siger at der findes et naturligt tal $y$, der er større end alle naturlige tal x. Dette er ikke rigtigt, da de naturlige tal ikke har en øvre grænse, og derfor er der uendelige naturlige tal som kan være større. 

2. Angiv negeringen af udsagnet 1. fra spørgsmål a) Negerings-operatoren (¬) må ikke indgå i dit udsagn.
\\

\begin{center}
    \textbf{Negationen af udsagn 1}  : $\exists y \in \mathbb{N}: \forall x \in \mathbb{N}: x \geq y$

\end{center}

Dette er negering af udsagnet 1. $\forall x \in \mathbb{N}: \exists y \in \mathbb{N}: x<y$, fordi det betyder at der findes et naturligt tal x, hvor der ikke er et tal y som er større end x. Symbolsk ville man skrive det som $\exists y \in \mathbb{N}: \forall x \in \mathbb{N}: x \geq y$