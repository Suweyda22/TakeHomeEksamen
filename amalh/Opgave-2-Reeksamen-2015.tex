\documentclass{article}
\usepackage{graphicx} % Required for inserting images

\input{Setup/Preamble}
\geometry{a4paper,top=30mm,bottom=30mm,right=35mm,left=25mm}
\setlength{\parindent}{0em}
\setlength{\parskip}{1em}
%\setlength\parindent{24pt}
\renewcommand{\baselinestretch}{1.5}
\renewcommand{\thesection}{\arabic{section}}
\renewcommand{\contentsname}{Indholdsfortegnelse}
\hypersetup{
    colorlinks=true,
    linkcolor=blue,
    filecolor=magenta,     
    urlcolor=cyan,
    citecolor=blue,
}
\renewcommand{\figurename}{Figur}
\renewcommand{\tablename}{Tabel}
\usepackage[format=plain,
            labelfont={bf,it},
            textfont=it]{caption}

           
\addbibresource{Bibliography.bib}
% Removing the bibliography heading
\defbibheading{myheading}[]{}
%Path relative to the main .tex file
\graphicspath{ {./Figures/} }

% Table of contents (TOC) and numbering of headings
\setcounter{tocdepth}{2}   
\setcounter{secnumdepth}{4}

\setcounter{biburllcpenalty}{9000}

 
%flowchart
\tikzstyle{startstop} = [rectangle, rounded corners, minimum width=3cm, minimum height=1cm, text width=3cm, text centered, draw=black, fill=red!30]
\tikzstyle{io} = [trapezium, trapezium left angle=70, trapezium right angle=110, minimum width=3cm, minimum height=1cm, text width=3cm, text centered, draw=black, fill=blue!30]
\tikzstyle{process} = [rectangle, minimum width=3cm, minimum height=1cm, text width=3cm, text centered, draw=black, fill=orange!30]
\tikzstyle{decision} = [diamond, aspect = 2, minimum width=3cm, minimum height=1cm, text width=3cm, text centered, draw=black, fill=green!30]

\tikzstyle{interupt} = [draw,rectangle split, rectangle split horizontal,rectangle split parts=3,minimum height=1cm,minimum width=3cm,draw=black,fill=yellow!30]
\tikzstyle{arrow} = [thick,->,>=stealth]

\usepackage{xcolor}

\definecolor{codegreen}{rgb}{0,0.6,0}
\definecolor{codegray}{rgb}{0.5,0.5,0.5}
\definecolor{codepurple}{rgb}{0.58,0,0.82}
\definecolor{backcolour}{rgb}{0.95,0.95,0.92}

\lstdefinestyle{mystyle}{
    basicstyle=\ttfamily,
    numbers=left,
    numberstyle=\tiny,
    numbersep=5pt,
    breaklines=true,
    captionpos=b,
    frame=single,
    lineskip=1.0ex, % reduced line spacing to 1.0
    language=Java, % or the language you are using 
    commentstyle=\color{codegreen},
    keywordstyle=\color{magenta},
    numberstyle=\tiny\color{codegray},
    stringstyle=\color{codepurple},
    otherkeywords={>,<,.,;,-,!,=,~},
    morekeywords={>,<,.,;,-,!,=,~}
}


\lstdefinestyle{Dockerfilestyle}{
    basicstyle=\ttfamily,
    numbers=left,
    numberstyle=\tiny,
    numbersep=5pt,
    breaklines=true,
    captionpos=b,
    frame=single,
    lineskip=1.0ex, % reduced line spacing to 1.0
}


\title{Expense tracker}
\author{Maysun Hassan, Suweyda Abdille, Amal Hassan}
\date{November 2024}

\begin{document}
\begin{titlepage}
\setstretch{1}
        \begin{center}
        \textsc{\LARGE SYDDANSK UNIVERSITET}\\[0.3cm]
                \textsc{\Large INSTITUT FOR MATEMATIK OG DATALOGI}\\[0.3cm]
                \textsc{\large }\\[1.2cm]

   
      \vspace{0.5cm}
        \textsc{\large{Fagnummer: DM500}}\\[0.5cm]
        \textsc{\large{Studieintroduktion til Datalogi og Kunstig Intelligens, 1. Semester}}\\[0.5cm]
        \textsc{\large{Maysun Hassan, Suweyda Abdille, Amal Hassan}}\\[0.5cm]
        % Title
        \rule{\linewidth}{0.5mm}\\[0.4cm]
        { \LARGE \bfseries  DM500: Take Home Eksamen \\[0.4cm]}
        \rule{\linewidth}{0.5mm}\\[1.5cm]
        % Authors and supervisor
         
       
        \vspace{1.5cm}

       
        % Bottom of the page  
        \textbf{Lærer:} Søren Sten Hansen \\ \mbox{}\\
        \textbf{Afleverings frist:} 18/11/2024
       
       \end{center}
       

\clearpage
   \end{titlepage}

\section*{Opgave 1}

\\
\\
I det følgende lader vi U = \{1, 2, 3, ..., 15\} være universet (universal set).
\\
Betragt de to mængder.
\begin{center}
    \( A = \{ 2n \mid n \in S \} \) og \( B = \{ 2n + 2 \mid n \in S \} \)
\end{center}
\\
hvor \(S = \{1, 2, 3, 4\}.\)
\\
Angiv Samtlige elementer i hver af følgende mængder.
\\
\\
\textbf{a) A}\\
Siden mængden A er defineret som \(\{ 2n \mid n \in S \}\), ved vi at vi kan indsætte elementerne fra mængden S på n's plads og tilføje dem til A, hvis resultatet er \(1 <= x <= 15\), hvilket vi ved fra vores univers.
\begin{center}
\(A = \{2*1, 2*2, 2*3, 2*4\} = \textbf{\{2, 4, 6, 8\}}\) \\
\end{center}
\\
\textbf{b) B}\\
Mængden B er defineret som \(B = \{2n + 2 \mid n \in S\}\), så vi kan igen indsætte de relevante elementer på de rigtige pladser. 
\begin{center}
    \(B = \{(3*1+2), (3*2+2), (3*3+2), (3*4+2)\} = \textbf{\{5, 8, 11, 14\}}\)
\end{center}
\textbf{c) A \(\cap\) B}\\
A intersection B refererer til mængden at de elementer som både er i A og B. Vi kan ud fra vores svar i del-opgaverne a) og b) se at det eneste der optræder i begge mængder er 8, hvilket giver os: 
\begin{center}
    \(A \cap B = \textbf{\{8\}}\) \\
\end{center}
\textbf{d) A \(\cup\) B}\\
A union B refererer til mængden af alle elementer i A og B, hvilket giver os: 
\begin{center}
    \(A \cup B = \textbf{\{2, 4, 5, 6, 8, 11, 14\}}\)
\end{center}
\textbf{e) A \( - \) B}\\
Mængdedifferensen mellem A og B refererer til mængden af elementer som optræder i A, men ikke B. Dette giver os mængden:
\begin{center}
    \(A - B = \textbf{\{2, 4, 6\}}\)
\end{center}
\textbf{f) \(\overline{A}\)} \\
Komplementet af A refererer til elementerne der optræder i universet, men \textit{ikke} er i A. Dette giver:
\begin{center}
    \(\overline{A} = \textbf{\{1, 3, 5, 7, 9, 10, 11, 12, 13, 14, 15\}}\)
\end{center}
\\
\\

\section*{Opgave 2}

1. Hvilke af følgende udsagn er sande?

\begin{center}
    1. $\forall x \in \mathbb{N}: \exists y \in \mathbb{N}: x<y$
\\
    2. $\forall x \in \mathbb{N}: \exists ! y \in \mathbb{N}: x<y$
\\
    3. $\exists y \in \mathbb{N}: \forall x \in \mathbb{N}: x<y$
    
\end{center}

1. $\forall x \in \mathbb{N}: \exists y \in \mathbb{N}: x<y$

Dette udsagn er \textbf{sandt}. Det siger at for alle naturlige tal $x$, eksistere der naturlige tal $y$ som er større end $x$. Hvilket er sandt da vi altid kan finde et naturligt tal $y$ som kan være større end $x$.

2. $\forall x \in \mathbb{N}: \exists ! y \in \mathbb{N}: x<y$

Dette udsagn er \textbf{falsk}. Det siger at for alle naturlige tal $x$, eksisterer der lige præcis kun et $y$, sådan at y er større end $x$. Hvilket ikke passer, fordi der findes y $\mathbb{N}$ som kan være større end x. 

3. $\exists y \in \mathbb{N}: \forall x \in \mathbb{N}: x<y$

Dette udsagn er \textbf{falsk}. Dette udsagn siger at der findes et naturligt tal $y$, der er større end alle naturlige tal x. Dette er ikke rigtigt, da de naturlige tal ikke har en øvre grænse, og derfor er der uendelige naturlige tal som kan være større. 

2. Angiv negeringen af udsagnet 1. fra spørgsmål a) Negerings-operatoren (¬) må ikke indgå i dit udsagn.
\\

\begin{center}
    \textbf{Negationen af udsagn 1}  : $\exists y \in \mathbb{N}: \forall x \in \mathbb{N}: x \geq y$

\end{center}

Dette er negering af udsagnet 1. $\forall x \in \mathbb{N}: \exists y \in \mathbb{N}: x<y$, fordi det betyder at der findes et naturligt tal x, hvor der ikke er et tal y som er større end x. Symbolsk ville man skrive det som $\exists y \in \mathbb{N}: \forall x \in \mathbb{N}: x \geq y$
\\
\\
\section{Opgave 3}
\\
Lad \(R, S\) og \(T\) være binære relationer på mængden \(\{1, 2, 3, 4\}\) \\
\\
\textbf{a) Lad R = \{(1, 1), (2, 1), (2, 2), (2, 4), (3, 1), (3, 3), (4, 1), (4, 4)\}.\\
Er R en partiel ordning?}\\
En relation er en partial ordning når den er refleksiv, antisymmetrisk og transitiv. \\
- Refleksivitet: alle elementer peger på sig selv. \\
- Antisymmetri: Hvis vi har \(a, b\) i relationen, må vi ikke have \(b, a\) - medmindre \(a = b\)\\
- Transitivitet: Hvis vi har \(a, b\) og \(b, c\) i relationen skal vi også have \(a, c\) \\
\\
Ud fra denne information kan vi nu afgøre om R er en partial ordning\\
\\
Refleksivitet: \((1, 1), (2, 2), (3, 3), (4, 4)\) er alle med i \(R\), hvilket gør den refleksiv\\
Antisymmetri: Selvom \((1, 1), (2, 2), (3, 3), (4, 4)\) er med, er \(R\) stadig antisymmetrisk, da vi i disse tilfælde har \(a = b\) \\
Transitivitet: \\
- Vi har \((2, 1)\) og \((1, 1)\) så vi burde have \((2, 1)\) som allerede er i \(R\). \\
- Vi har \((2, 4)\) og \((4, 1)\) så vi burde have \((2, 1)\) som allerede er i \(R\). \\
- Vi har \((3, 1)\) og \((1, 1)\) så vi burde have \((3,1)\) som allerede er i \(R\). \\
- Vi har \((3, 3)\) og \((3,1)\) så vi burde have \((3,1)\) som allerede er i \(R\). \\
- Vi har \((4, 1)\) og \((1, 1)\) så vi burde have \((4, 1)\) som allerede er i \(R\). \\
Parene der peger på sig selv opfylder selvfølgelig også alle transitiviteten.
\begin{center}
\(R\) opfylder alle krav og er derfor \textbf{en partial ordning}
\end{center}
\\
\textbf{b) Lad S = \{(1,2), (2, 3), (2, 4), (4, 2)\}.\\
Angiv den transitive lukning af S.}\\
\\
Vi har allerede defineret hvad transitivitet er. For at finde den transitive lukning af S skal vi kigge gennem de forskellige par, og kontrollere om de opfylder transitivitet. Hvis de ikke gør, skal vi tilføje det par, som gør dem transitive.
\\
- Vi har \((1, 2)\) og \((2, 3)\) så vi burde have \((1, 3)\).
- Vi har \((1, 2)\) og \((2, 4)\) så vi burde have \((1, 4)\).
- Vi har \((4, 2)\) og \((2, 3)\) så vi burde have \((4, 3)\).
- Vi har \((4, 2)\) og \((2, 4)\) så vi burde have \((4, 4)\).
\\
Hvis vi så sætter det hele sammen får vi den transitive lukning af S som: \\
\begin{center}
\(= \{(1, 2), (2, 3), (2, 4), (4, 2), (1, 3), (1, 4), (4, 3), (4, 4)\}\)
\end{center}
\\



\textbf{C.) Lad T = \{(1,1), (1,3), (2,2) (2,4), (4,2), (3,1), (3,3), (4,2), (4,4)\}} .
\\
Bemærk, at T er en ævivalens-relation. 
\\
\textbf{Angiv T's ækvivalens-klasser.}
\\ 
En ækvivalens relation er en relation som er refleksiv, symmetrisk og transitiv. \\

- Refleksivitet: alle elementer peger på sig selv \((1,1)\) og \((2,2)\) og \((3,3)\) og \((4,4)\). \\
- Antisymmetri: Hvis vi har \((a, b)\) i relationen, skal vi have \((b, a)\) med. \\
- Transitivitet: Hvis vi har \((a, b)\) og \((b, c)\) i relationen skal vi også have \((a, c)\) 

\\ 
Ækvivalensklasser er en gruppe af elementer som er relateret til hianden igennem en ækvivalensrelation. som opfylder de tidligere egenskaber.
\\
I relationen T, har vi mængden {1,2,3,4}.\\
Hvor 1 er relateret til 3 via \((1,3)\) og \((3,1)\) 
Derfor er 1 og 3 i samme ækvivalensklasse\\
Hvor 2 er relateret til 4 via \((2,4)\) og \((4,2)\) 
Derfor er 2 og 4 også i samme ækvivalensklasse. 
\\
De er de eneste ækvivalente par i relationen T, som ikke er relateret til sig selv.\\
Derfor er T's ækvivalensklasser:
\begin{center}
    \(= \{(1, 3), (2, 4)\}\)
\end{center}
\\
\\
\section*{Opgave 3} 2009
Lad S = \{1, 2, \ldots, 15\}. 
Betragt følgende binære relation på \( S \):
\[
R = \{(a, b) \mid b = 2a\}
\]
\begin{enumerate}
    \item[a)] Hvilke af nedenstående par tilhører \( R \)? Hvilke tilhører \( R^2 \)?
    \[
    (1, 1), \, (2, 4), \, (4, 2), \, (3, 5), \, (2, 8)
    \]
    \item[b)] Opskriv alle par i den transitive lukning af \( R \).
\end{enumerate}
Betragt relationen \( R = \{(a, b) \mid b = 2a\} \) på \( S = \{1, 2, 3, 4, 5, 6\} \).
\textbf{Matrixrepræsentation af \( R \):}
Vi undersøger alle mulige par \((a, b)\), hvor \( a, b \in S \), og sætter \( R_{ij} = 1 \), hvis \( b = 2a \), ellers \( R_{ij} = 0 \).
\[ 
R =
\begin{bmatrix}
0 & 1 & 0 & 0 & 0 & 0 \\ 
0 & 0 & 1 & 0 & 0 & 0 \\ 
0 & 0 & 0 & 1 & 0 & 0 \\ 
0 & 0 & 0 & 0 & 1 & 0 \\ 
0 & 0 & 0 & 0 & 0 & 1 \\ 
0 & 0 & 0 & 0 & 0 & 0
\end{bmatrix}
\]
\textbf{Forklaring:}
\begin{itemize}
    \item \((1, 2)\): \( b = 2 \cdot 1 \), så \( R_{1,2} = 1 \).
    \item \((2, 4)\): \( b = 2 \cdot 2 \), så \( R_{2,3} = 1 \).
    \item \((3, 6)\): \( b = 2 \cdot 3 \), så \( R_{3,4} = 1 \).
    \item \((4, 8)\), \((5, 10)\), \((6, 12)\): Ikke i \( S \), så deres indgange er 0.
\end{itemize}

\thispagestyle{empty}
\newpage
\setcounter{page}{1}

\end{document}