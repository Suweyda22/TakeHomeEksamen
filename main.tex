\documentclass{article}
\usepackage{graphicx} % Required for inserting images

\input{Setup/Preamble}
\geometry{a4paper,top=30mm,bottom=30mm,right=35mm,left=25mm}
\setlength{\parindent}{0em}
\setlength{\parskip}{1em}
%\setlength\parindent{24pt}
\renewcommand{\baselinestretch}{1.5}
\renewcommand{\thesection}{\arabic{section}}
\renewcommand{\contentsname}{Indholdsfortegnelse}
\hypersetup{
    colorlinks=true,
    linkcolor=blue,
    filecolor=magenta,     
    urlcolor=cyan,
    citecolor=blue,
}
\renewcommand{\figurename}{Figur}
\renewcommand{\tablename}{Tabel}
\usepackage[format=plain,
            labelfont={bf,it},
            textfont=it]{caption}

           
\addbibresource{Bibliography.bib}
% Removing the bibliography heading
\defbibheading{myheading}[]{}
%Path relative to the main .tex file
\graphicspath{ {./Figures/} }

% Table of contents (TOC) and numbering of headings
\setcounter{tocdepth}{2}   
\setcounter{secnumdepth}{4}

\setcounter{biburllcpenalty}{9000}

 
%flowchart
\tikzstyle{startstop} = [rectangle, rounded corners, minimum width=3cm, minimum height=1cm, text width=3cm, text centered, draw=black, fill=red!30]
\tikzstyle{io} = [trapezium, trapezium left angle=70, trapezium right angle=110, minimum width=3cm, minimum height=1cm, text width=3cm, text centered, draw=black, fill=blue!30]
\tikzstyle{process} = [rectangle, minimum width=3cm, minimum height=1cm, text width=3cm, text centered, draw=black, fill=orange!30]
\tikzstyle{decision} = [diamond, aspect = 2, minimum width=3cm, minimum height=1cm, text width=3cm, text centered, draw=black, fill=green!30]

\tikzstyle{interupt} = [draw,rectangle split, rectangle split horizontal,rectangle split parts=3,minimum height=1cm,minimum width=3cm,draw=black,fill=yellow!30]
\tikzstyle{arrow} = [thick,->,>=stealth]

\usepackage{xcolor}

\definecolor{codegreen}{rgb}{0,0.6,0}
\definecolor{codegray}{rgb}{0.5,0.5,0.5}
\definecolor{codepurple}{rgb}{0.58,0,0.82}
\definecolor{backcolour}{rgb}{0.95,0.95,0.92}

\lstdefinestyle{mystyle}{
    basicstyle=\ttfamily,
    numbers=left,
    numberstyle=\tiny,
    numbersep=5pt,
    breaklines=true,
    captionpos=b,
    frame=single,
    lineskip=1.0ex, % reduced line spacing to 1.0
    language=Java, % or the language you are using 
    commentstyle=\color{codegreen},
    keywordstyle=\color{magenta},
    numberstyle=\tiny\color{codegray},
    stringstyle=\color{codepurple},
    otherkeywords={>,<,.,;,-,!,=,~},
    morekeywords={>,<,.,;,-,!,=,~}
}


\lstdefinestyle{Dockerfilestyle}{
    basicstyle=\ttfamily,
    numbers=left,
    numberstyle=\tiny,
    numbersep=5pt,
    breaklines=true,
    captionpos=b,
    frame=single,
    lineskip=1.0ex, % reduced line spacing to 1.0
}


\title{Expense tracker}
\author{Maysun Hassan, Suweyda Abdille, Amal Hassan}
\date{November 2024}

\begin{document}
\begin{titlepage}
\setstretch{1}
        \begin{center}
        \textsc{\LARGE SYDDANSK UNIVERSITET}\\[0.3cm]
                \textsc{\Large INSTITUT FOR MATEMATIK OG DATALOGI}\\[0.3cm]
                \textsc{\large }\\[1.2cm]

   
      \vspace{0.5cm}
        \textsc{\large{Fagnummer: DM500}}\\[0.5cm]
        \textsc{\large{Studieintroduktion til Datalogi og Kunstig Intelligens, 1. Semester}}\\[0.5cm]
        \textsc{\large{Maysun Hassan, Suweyda Abdille, Amal Hassan}}\\[0.5cm]
        % Title
        \rule{\linewidth}{0.5mm}\\[0.4cm]
        { \LARGE \bfseries  DM500: Take Home Eksamen \\[0.4cm]}
        \rule{\linewidth}{0.5mm}\\[1.5cm]
        % Authors and supervisor
         
       
        \vspace{1.5cm}

       
        % Bottom of the page  
        \textbf{Lærer:} Søren Sten Hansen \\ \mbox{}\\
        \textbf{Afleverings frist:} 18/11/2024
       
       \end{center}
       

\clearpage
   \end{titlepage}

\addtocontents{toc}{\protect\thispagestyle{empty}}
\tableofcontents
\thispagestyle{empty}
\newpage
\setcounter{page}{1}
\section{Introduktion} 

I denne opgave er det primære mål at dokumentere, at man kan bruge Latex
og Git. Opgaven skal løses i ens studiegruppe, og man skal bruge Git til at
udvikle en Latex-fil sammen i gruppen. Opgaven evalueres med bestået/ikke
bestået. Kravet for beståelse er:
1. Afleveringen er lavet i Latex.
2. Der er brugt Git under udviklingen, alle personer bag afleveringen har
lavet mindst et Git-commit hver, og det samlede antal linier committed
fra hver person er nogenlunde jævnt fordelt.
3. Afleveringen indeholder (seriøse forsøg på) løsninger til det krævede
antal opgaver (se nedenfor). 
\section{Opgave 1}

I det følgende lader vi U = {1,2,3,...,15} være universet (universal set). Betragt de to mængder
A = 2n | n ∈ S og B = 3n + 2 | n ∈ S
hvor S = {1,2,3,4}.
Angiv samtlige elementer i hver af følgende mængder
a. A 
b. B
c. A og B 
d. A eller B 
e. A - B
f. A 

\section*{Opgave 1}

\\
\\
I det følgende lader vi U = \{1, 2, 3, ..., 15\} være universet (universal set).
\\
Betragt de to mænger
\\
\\
\begin{center}
    \( A = \{ 2n \mid n \in S \} \) og \( B = \{ 2n + 2 \mid n \in S \} \)
\end{center}
\\
\\
hvor \(S = \{1, 2, 3, 4\}.\)
\\
Angiv Samtlige elementer i hver af følgende mængder
\\
\\
a) A\\
Siden mængden A er defineret som \(\{ 2n \mid n \in S \}\), ved vi at vi kan indsætte tallene fra mængden S på n's plads og tilføje dem til A, hvis resultatet er \(1 <= x <= 15\), hvilket vi ved fra vores univers.
\\
\\
\(A = \{2*1, 2*2, 2*3, 2*4\} = \textbf{\{2, 4, 6, 8\}}\)
\\
\\
b) B\\
c) \(A \cap B\)\\
d) \(A \cup B\)\\
e) \(A - B\)\\
f) \(\overline{A}\)
\\
\\

\section{Opgave 2}

Hvilke af følgende udsagn er sande?
1. ∀x∈N:∃y∈N:x<y 
2. ∀x∈N:∃!y∈N:x<y 
3. ∃y∈N:∀x∈N:x<y

Angiv negeringen af udsagn nummer et fra spørgsmålet før
a). Negerings-operatoren (¬) må ikke indgå i dit udsagn.
\section{Opgave 3}

Lad R, S og T være binære relationer på mængden {1, 2, 3, 4}.
a) Lad R = {(1,1),(2,1),(2,2),(2,4),(3,1),(3,3),(3,4),(4,1),(4,4)}.
Er R en partiel ordning?

b) Lad S = {(1, 2), (2, 3), (2, 4), (4, 2)}.
Angiv den transitive lukning af S.

c) Lad T = {(1,1),(1,3),(2,2),(2,4),(3,1),(3,3),(4,2),(4,4)}.
Bemærk, at T er en ækvivalens-relation. Angiv T's ækvivalens-klasser.

\section{Opgave 3 2009}

Lad S = {1, 2, . . . , 15}. 
Betragt følgende binære relation på S: R = {(a, b) | b = 2a}
a) Hvilke af nedenstående par tilhører R? Hvilke tilhører R2? 
(1, 1), (2, 4), (4, 2), (3, 5), (2, 8)
b) Opskriv alle par i den transitive lukning af R.


\end{document}
